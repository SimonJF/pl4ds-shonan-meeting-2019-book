%%%%%%%%%%%%%%%%%%%%%%%%%%%%%%%%%%%%%%%%%%%%%%%%%%%%%%%
%
% chapter04 - Consistency Models
%
%%%%%%%%%%%%%%%%%%%%%%%%%%%%%%%%%%%%%%%%%%%%%%%%%%%%%%%

%
% >>>>>>>>>>>>>>> PLEASE NOTE <<<<<<<<<<<<<<<
%
% This file is not stand-alone compileable as it is, to make it compileable while writing uncomment the preamble below.
% In this case, you also have to uncomment the begin/end document statements.
% You can outcomment the preamble and the begin/end document statements again or erase them when handing in your contribution.
%
% If you use BibTex for your bibliography, please use \putbib[bibliography] to print your reference (see end of this file).
%
% you can use paths relative to your chapter dir, e.g. \figure{assets/fig1}.
%
% >>>>>>>>>>>>>>>>>>>><<<<<<<<<<<<<<<<<<<<<<<

%%%%%%%%%%%%%%%%%%%%%%%%%%%%%%%%%%%%%%%%%%%%%%%%%%
%% you can uncomment the following preamble during development to make this file compileable.
%% Note that you need the svmult.cs file inside your chapter root dir to make this work.
%% Also note that if you need additional packages etc., you can add them here, but please
%% mark them somehow so the editor of this book knows you need them in the final book.
%% When you hand in your contribution, please uncomment or remove the preamble again.
%%%%%%%%%%%%%%%%%%%%%%%%%%%%%%%%%%%%%%%%%%%%%%%%%%
%%%%%%%%%%%%%%%%%%%%%%%%%%%%%%%%%%%%%%%%%%%%%%%%%%% start of preamble
%\documentclass[
%graybox,
%envcountchap,
%%natbib
%]{svmult}
%
%\usepackage[utf8]{inputenc}
%%\usepackage{type1cm}        % activate if the above 3 fonts are 
%% not available on your system
%
%\usepackage{makeidx}         % allows index generation
%\usepackage{graphicx}        % standard LaTeX graphics tool
%% when including figure files
%\usepackage{multicol}        % used for the two-column index
%\usepackage[bottom]{footmisc}% places footnotes at page bottom
%
%\usepackage{newtxtext}       % 
%\usepackage{newtxmath}       % selects Times Roman as basic font
%
%% \usepackage{natbib}
%\usepackage{footmisc}
%
%%% Additional packages added. Add necessary packages here.
%%\usepackage[english]{babel}
%\usepackage{siunitx}
%\usepackage{amssymb}
%\usepackage{pifont}
%\usepackage{xcolor}
%\usepackage{tabularx}
%\usepackage{listings}
%\usepackage{booktabs}
%\usepackage{hyperref}
%\usepackage{url}
%\usepackage{mathtools}
%\usepackage{lipsum}
%\usepackage{import}
%\usepackage{bibunits}
%\usepackage{acronym}
%\usepackage[nottoc]{tocbibind}
%\usepackage{numberpt}
%
%\newcommand*{\CHAPTERSROOT}{../.}	% root path for chapters.
%\newcommand*{\chapterprefix}{04}	% your chapter number.
%
%\makeindex % used for the subject index
%%%%%%%%%%%%%%%%%%%%%%%%%%%%%%%%%%%%%%%%%%%%%% end of preamble

%% uncomment the \begin{document} statement to make this file stand-alone compileable.
%\begin{document}

\begin{bibunit}
	
	\title*{Consistency Models}
	\author{Carlos Baquero and Carla Ferreira}
	
	\institute{
		Carlos Baquero \at University of Minho, Portugal, \email{cbm@di.uminho.pt}
		\and Carla Ferreira \at NOVA University Lisbon, Portugal, \email{carla.ferreira@fct.unl.pt}
		\and Sebastian Burckhardt \at Microsoft Research, USA, \email{sburckha@microsoft.com}
		\and D\'{e}fago Xavier \at Tokyo Institute of Technology, Japan, \email{defago@c.titech.ac.jp}
		\and Philipp Haller \at KTH Royal Institute of Technology, Sweden, \email{phaller@kth.se}
		\and Elisa Gonzalez Boix \at Vrije Universiteit Brussel, Belgium, \email{egonzale@vub.ac.be}
	}
	\maketitle
	
	\abstract{Please place your abstract here.}

        \section{Introduction}

        The following is the result of a focus group on consistency models. From the title the reader could expect a comprehensive survey of the subject. Instead, our goal is less ambitious and only to provide a vision of some of the dimensions that characterize this area and then offer the reader some open research directions. For a deeper coverage of the field we direct the reader to surveys on \emph{non-transactional consistency models} \cite{Viotti:2016:CND:2911992.2926965}, ...  %TODO: to complete with other reference surveys by sub-area: database consistency, memory consistency models, distributed consensus and total order

        (a couple more paragraphs on the focus of our discussion)


	\section{Consistency at different scales}\label{sec:1}

        %Relevant notes from meeting:
        % - Issue in both multi-core and distributed settings
        % -- Relation between two areas
        % - In distributed systems, strong consistency has problem with latency:
        % -- What are good timeouts?
        % -- Not much control over rate of commits
        % - Eventual consistency:
        % -- Asynchronicity an issue
        % - Shared-memory systems: sync operation to establish consistent view
        % - Essential differences between multi-core and distributed setting
        % -- Partial failure
        % -- Unbounded asynchronicity

        \section{The two sides of the CAP frontier}\label{sec:2}

        %Relevant notes from meeting:
        % - Escrow is an intermediate point between weak consistency and strong consistency
        % -- Escrow: replicas can coordinate in pairs to re-fill their credits
        % -- Consensus round only needed for strong consistency
        % - Best you can do under partition tolerance: causal consistency

        \section{Walks around the frontier}\label{sec:2}

        %Relevant notes from meeting:
        % - Semantic causal consistency
        % -- Relax consistency for independent data
        % - Atomicity
        % -- Transactions/bundles of operations
        % -- Simple to do under partition tolerance

        \section{Research Directions}\label{sec:2}

        %Relevant notes from meeting:
        % - How to provide causal consistency in a scalable way
        % - How to program with different consistency levels?
        % -- Suresh Jagannathan et al. define consistency in terms of operation order
        % --- can be hard to reason about
        % - Providing both weak and strong operations. How to describe in the language
        % - Debugging of distributed applications built on replicated data types
        % -- Consistency bugs
        % - Bounded staleness (probabilistic bounds)
        % - Composition of replicated data types
        % -- How to ensure consistency?
        % - Fragility at the edge to the clients
        % -- Raft leader receives request, but connection to client breaks
        % -- Client cannot be part of replicas
        % - Security and privacy
        % -- Encryption, (semi-)homomorphic encryption
        % --- Certain operations performed on encrypted data
        % - Dynamic Membership
        % - Partial replication
	
	%% content
	%\section{Content}\label{sec:1}
	
	%Please place your content here. See the sample template chapter for further reference on how you could style your content here.
	
	%% content
	
	%\section*{Appendix}\label{appendix}
	
	%Please place your appendix content here, if applicable.

	
	%%%%%%%%%%%%%%%%%%%%%%%%%%%%%%%%%%%%%%%%%%%%%%%%%%%%%%%%%%%%%%%%%%%%%%%%%%%%%%%%%%%%%%%%%%%%%%%%%%%%%%%%
	%% For your bibliography, you should use a bibtex .bib file and include it here.
	%% Note that the final reference lists styling might differ because it'll be styled in unified book layout.
	
	% \biblstarthook{
	%	text inserted here will be printed before the actual list of references, but only if there is at least one reference to %display. Delete this section if you don't need it.
	%}
	
	% \nocite{*}		%% uncomment if uncited references should be listed in the bibliography.
	
	%% uncomment and state path to your .bib to use a bibtex file as your bibliography.
	%% NOTE: relative paths don't work in \putbib => During development, you might delete the "\CHAPTERSROOT/chapter\chapterprefix/" part to refer to your bib file. When you're done, please make this path absolute by adding the prefix again.
	%%
	% \putbib[\CHAPTERSROOT/chapter\chapterprefix/bibliography] %
	
\end{bibunit}
	
%% uncomment the \end{document} statement to make this file stand-alone compileable.
%\end{document}
