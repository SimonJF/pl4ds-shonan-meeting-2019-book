\begin{figure}
\centering
\subfloat[Channel\label{fig:bt:channels-vs-mailboxes:a}]{
  \centering
  \begin{tikzpicture}

    \matrix (m) [matrix of math nodes,row sep=2em,column sep=7em,minimum
    width=1em,ampersand replacement=\&]
    {
       P_1 \& P_1 \\
       P_2 \& P_2 \\
       P_3 \& P_3 \\
    };
    \path
      (m-1-1) edge (m-1-2)
              edge (m-2-2)
              edge (m-3-2)
      (m-2-1) edge (m-1-2)
              edge (m-2-2)
              edge (m-3-2)
      (m-3-1) edge (m-1-2)
              edge (m-2-2)
              edge (m-3-2);

    %% \tikzset{dotted/.style={draw=black!50!white, line width=0.5pt,
    %%                         dash pattern=on 2pt off 2pt,
    %%                         inner sep=0mm, rectangle, rounded corners}};

    \node (sender) [fit = (m-1-1) (m-3-1)] {};
    \node at (sender.north) [above] {\textsf{sender}};

    \node (receiver) [fit = (m-1-2) (m-3-2)] {};
    \node at (receiver.north) [above] {\textsf{receiver}};

  \end{tikzpicture}
}
~
\subfloat[Mailbox\label{fig:bt:channels-vs-mailboxes:b}]{
  \centering
  \begin{tikzpicture}
    \matrix (m) [matrix of math nodes,row sep=2em,column sep=7em,minimum
    width=1em,ampersand replacement=\&]
    {
       P_1 \& P_1 \\
       P_2 \& P_2 \\
       P_3 \& P_3 \\
    };
    \path
      (m-1-1) edge (m-2-2)
      (m-2-1) edge (m-2-2)
      (m-3-1) edge (m-2-2);

    \node (sender) [fit = (m-1-1) (m-3-1)] {};
    \node at (sender.north) [above] {\textsf{sender}};

    \node (receiver) [fit = (m-1-2) (m-3-2)] {};
    \node at (receiver.north) [above] {\textsf{receiver}};

  \end{tikzpicture}
}

\caption{Mailboxes as pinned channels}
\label{fig:bt:channels-vs-mailboxes}
\end{figure}
